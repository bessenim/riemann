\documentclass[%
master,      % тип документа
subf,        % использовать пакет subcaption для вложенной нумерации рисунков
href,        % использовать пакет hyperref для создания гиперссылок
%colorlinks,  % цветные гиперссылки
%fixint,     % включить прямые знаки интегралов
]{disser}

\usepackage[
  a4paper, mag=1000,
  left=3cm,right=2cm,top=2cm,bottom=2cm]{geometry}

\usepackage[intlimits]{amsmath}
\usepackage{amssymb,amsfonts}
\usepackage{multirow}
\usepackage{url}
\usepackage{verbatim} 

\usepackage[T2A]{fontenc}
\usepackage[utf8]{inputenc}
\usepackage[english,russian]{babel}
\ifpdf\usepackage{epstopdf}\fi
\usepackage[autostyle]{csquotes}

\bibliographystyle{gost2008}

% Шрифт Times в тексте как основной
%\usepackage{tempora}
% альтернативный пакет из дистрибутива TeX Live
%\usepackage{cyrtimes}

% Шрифт Times в формулах как основной
%\usepackage[varg,cmbraces,cmintegrals]{newtxmath}
% альтернативный пакет
%\usepackage[subscriptcorrection,nofontinfo]{mtpro2}

%\usepackage[style=gost-numeric,backend=biber,language=auto,hyperref=auto,autolang=other,sorting=none]{biblatex}


% Номера страниц снизу и по центру
\pagestyle{footcenter}
\chapterpagestyle{footcenter}

% Точка с запятой в качестве разделителя между номерами цитирований
%\setcitestyle{semicolon}

% Использовать полужирное начертание для векторов
\let\vec=\mathbf

% Включать подсекции в оглавление
\setcounter{tocdepth}{2}

\graphicspath{{fig/}}

\DeclareUnicodeCharacter{2060}{\nolinebreak}

\begin{document}
\institution{Министерство образования и науки Российской Федерации\\ 
Федеральное государственное автономное образовательное учреждение\\
высшего профессионального образования\\
«Московский физико-технический институт\\
(государственный университет)»\\
\bigskip 
Факультет Проблем Физики и Энергетики\\
\bigskip 
Кафедра фундаментальных взаимодействий и космологии\\ 
	}



\title{Трансмутация ядер свинца в столкновениях встречных пучков на БАК}
\topic{Выпускная квалификационная работа\\(магистерская диссертация)}


% Автор
\author{У.А.~Дмитриева}
% Группа
\group{283}
% Номер направления
\coursenum{03.04.01}
\course{«Прикладные математика и физика»}
% Номер магистерской программы
%\masterprognum{111111}
\masterprog   {«Физика фундаментальных взаимодействий»}

% Научный руководитель
\sa      { И.А.~Пшеничнов}
\sastatus{д.~ф.-м.~н., в.~н.~с.}

% Рецензент
\rev      {ФИО рецензента}
\revstatus{д.~ф.-м.~н., в.~н.~с.}
% Второй рецензент
%\revsnd      {ФИО рецензента}
%\revsndstatus{д.~т.~н., ст.~н.~с.}

% Город и год
\city{Москва}
\date{\number\year}

\maketitle
		

% Переопределение стандартных заголовков
%\def\contentsname{Содержание}
%\def\conclusionname{Выводы}
%\def\bibname{Литература}


% Содержание
\tableofcontents

% Введение
\input{Diplom-Introduction}

%\input{Diplom-Discussion}
%\input{Diplom-Conclusion}
$\pmb{\mathscr{S}}\left( {{R^N}} \right)$.

% Список литературы
\newpage
\bibliography{Diploma}


% Приложения
\appendix
%\input{Diplom-Attached}

\end{document}


