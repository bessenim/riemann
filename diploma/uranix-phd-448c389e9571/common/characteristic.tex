\ifdisser
\newcommand{\actuality}{{\textbf{Актуальность темы.}}}
\newcommand{\develop}{{\textbf{Степень разработанности темы.}}}
\else
\newcommand{\actuality}{{\textbf{Актуальность и степень разработанности темы.}}}
\fi
\newcommand{\aim}{{\textbf{Целями}}}
\newcommand{\tasks}{{\textbf{задачи}}}
\newcommand{\novelty}{{\textbf{Научная новизна:}}}
\newcommand{\influence}{{\textbf{Теоретическая и практическая значимость работы:}}}
\newcommand{\methodology}{{\textbf{Методология и методы исследования.}}}
\newcommand{\defpositions}{{\textbf{Положения, выносимые на~защиту}}}
\ifdisser
\newcommand{\probation}{{\textbf{Степень достоверности и апробация результатов работы.}}}
\else
\newcommand{\probation}{{\textbf{Степень достоверности и апробация работы.}}}
\fi

{\actuality} Перенос излучения является существенным процессом в разнообразных задачах динамики высокотемпературной плазмы.
%Сильноточные разряды в установках на основе Z-пинч  
При пропускании сильноточного (порядка нескольких мегаампер) разряда через проволочную сборку образуется плазменный шнур~"--- пинч,
являющийся мощным источником рентгеновского излучения, которое может быть использовано, в частности, для экспериментального исследования процессов, протекающих в плазме с высокой плотностью и энергией. 
В астрофизике актуальной задачей является моделирование различных объектов, например, аккреционных дисков молодых звезд, квазаров, релятивистских струй и т. п.
Для этих задач судить об адекватности математической модели можно сравнивая наблюдаемый спектр объекта со спектром, получаемым в численных расчетах.
Перенос излучения важен и в метеорологии, так как он играет ключевую роль в атмосферном теплообмене и, как следствие, в формировании климата планеты.

Для решения уравнения переноса излучения разработаны разнообразные методы, учитывающие ту или иную симметрию (плоская, цилиндрическая или сферическая симметрия задачи), различные свойства коэффициента поглощения (приближения серой материи, оптически тонкого и оптически толстого слоя) и приближенные способы описания угловой зависимости излучения (приближение <<вперед-назад>>, диффузионное приближение). Высокой точности численного решения можно достичь при условии локализации источников излучения, например, на границе расчетной области (метод дискретизации <<от границы>>). 

Хотя эти методы могут быть приемлемы в гидродинамическом моделировании, их результатов может быть недостаточно для сравнения с имеющимися экспериментальными данными. Например, в задачах моделирования Z-пинчей использование диффузионного приближения позволяет корректно учесть количество энергии, унесенной излучением. Но данное приближение не позволяет с достаточной точностью восстановить угловое распределение интенсивности излучения, а лишь интегральные характеристики, такие как плотность и поток энергии излучения, но не угловое распределение излучения. Это приводит к необходимости построения более точных методов решения уравнения переноса излучения.

С развитием вычислительной техники, в особенности, использованием графических ускорителей, становится возможным решать задачи переноса излучения в полноценной трехмерной постановке с достаточно подробным описанием частотной и угловой зависимости решения. В рамках приближений, принятых в данной работе, уравнение переноса излучения обладает существенным запасом параллелизма, поскольку решения вдоль различных направлений и на различных частотах не зависят друг от друга. Актуальной задачей является создание универсального программного блока, пригодного для работы совместно с различными существующими гидродинамическими программными комплексами. При этом существенно ограничивается свобода выбора таких характеристик, как тип расчетной сетки, способ ее декомпозиции на вычислительные подобласти.

\ifdisser
{\develop} Уравнение переноса излучения может быть рассмотрена как частный случай уравнения переноса частиц. Многие методы, построенные для задач нейтронного переноса, напрямую применимы к задаче переноса излучения. Отличие между этими задачами заключается в интегральном слагаемом, описывающем рассеяние частиц.

Для одномерных геометрий простейшими методами решения уравнения переноса является метод дискретных ординат \cite{Wick1943,Chandrasekar1950}, заключающийся в решении уравнения переноса вдоль дискретного набора направлений, на который разбивается диапазон возможных направлений полета фотона $\theta \in [0, \pi]$. Развитием этого метода для многомерного случая является $S_n$ метод \cite{Carlson1953,Lathrop1965}, в котором на участки разбивается вся сфера направлений.
Для цилиндрической и сферической симметрий был предложен ряд сеточно-характеристических методов \cite{Vladimirov1958,Goldin1960}. Особенностью криволинейных геометрий является наличие в уравнении переноса не только пространственных, но и угловых производных.

Широкое распространение получил метод, называемый диффузионным приближением. Он получается в предположении, что поток излучения направлен против градиента плотности энергии излучения. Популярность данного метода обусловлена тем, что он сводит решение кинетического уравнения переноса к решению одного скалярного уравнения эллиптического типа. Метод диффузионного приближения является простейшим приближением $P_n$ метода сферических гармоник \cite{Marshak1947,vladimirov1961,Devison1960}.

Развитием диффузионного метода является метод квазидиффузии \cite{Goldin1964}, в котором решение диффузионного уравнения уточняется введением тензора квазидиффузии, вычисляемом из решения уравнения переноса методом дискретных ординат.

В астрофизических приложениях для решения уравнения переноса излучения часто применяется метод длинных характеристик \cite{vladimirov1961}, основанный на точном решении уравнения переноса вдоль некоторого набора направлений. Использование данного метода порождает сильный <<эффект луча>>, то есть распространение излучения только в избранных направлениях. В этом случае адаптивный выбор набора направлений \cite{Galanin2010} для каждой пространственной точки может решить проблему. 
\fi

 \aim\ данной работы являются:
\begin{enumerate}
  \item Построение и исследование численных методов решения уравнения переноса излучения.
  \item Реализация полученных методов с использованием графических ускорителей.
  \item Моделирование линейчатого спектра излучения звезды типа Т Тельца при наличии конического ветра. 
\end{enumerate}

Для~достижения поставленных целей были решены следующие {\tasks}:
\begin{enumerate}
  \item На основании вариационного принципа Владимирова построен численный метод решения уравнения переноса излучения для произвольного базиса угловых функций.
  \item Построена квадратурная формула для полусферы и разработан метод численного построения квадратурных формул продолжением по параметру.
  \item Изучены вопросы сходимости метода для базиса из сферических функций и базиса из радиальных функций.
  \item Создана параллельная программная реализация метода, позволяющая проводить расчеты на графическом ускорителе.
  
  \item Построен маршевый алгоритм решения уравнения переноса излучения для неструктурированных тетраэдральных сеток методом коротких характеристик.
  \item Построен маршевый метод второго порядка аппроксимации и предложен способ монотонизации схемы.
  \item Показана корректность данного алгоритма в случае использования тетраэдральной сетки, удовлетворяющей условию Делоне.
  \item Предложен алгоритм упорядочения сеточных элементов для сеток, не удовлетворяющих условию Делоне. Предложено приложение алгоритма упорядочения для ярусно-параллельной формы графа зависимостей вычислительного алгоритма маршевого метода.
  
  \item Построен распределенный метод для решения уравнения переноса излучения, использующий длинные характеристики, ограниченные расчетной подобластью.
  \item Выведено трассировочное соотношение и доказана лемма об устойчивой трассировке.
  \item Метод реализован в многопроцессорном MPI-варианте, а также в варианте, использующем кластер из графических ускорителей.
  \item Изучены вопросы ускорения и эффективности параллельной реализации.
  
  \item Построена локально термодинамически равновесная математическая модель поглощения излучения околозвездным веществом, учитывающая доплеровский сдвиг частоты поглощения и различную заселенность уровней атома водорода.
  \item По результатам гидродинамического моделирования звезды типа Т Тельца проведен расчет спектра излучения в линии $\text{H-}\alpha$ при различных ориентациях плоскости акреционного диска.
\end{enumerate}

\novelty
\begin{enumerate}
  \item Впервые предложен вариационный метод для решения самосопряженного уравнения переноса излучения с базисом из радиальных угловых функций.
  \item Впервые предложен маршевый алгоритм для решения уравнения переноса на неструктурированной тетраэдральной сетке.
  \item Предложена оригинальная распределенная многопроцессорная реализация метода длинных характеристик.
\end{enumerate}

\influence\ Предложены и исследованы новые методы решения уравнения переноса излучения. Представлены алгоритмы распараллеливания предложенных методов. Оценена скорость сходимости методов по пространственным и угловым переменным.

Предложенные алгоритмы упорядочения неструктурированных третраэдральных сеток могут применяться для решения других стационарных гиперболических задач как конечно-разностными, так и конечно-объёмными численными методами.

Разработанные программные модули можно использовать в существующих гидродинамических программных комплексах, в том числе реализованных для кластерных вычислительных систем.

Работа выполнялась при поддержке проекта Министерства образования и науки РФ №3.522.2014/К <<Исследование процессов, происходящих в веществе, и изменения его свойств при импульсном вводе энергии высокой плотности>>.

\methodology\ Численные методы построены на основании вариационного подхода Ритца и различных сеточно-характеристических подходов. Для решения систем линейных уравнений использованы как итерационный метод сопряженных градиентов, так и прямые методы разложения разреженной матрицы. 
В основу распределенного метода положен принцип геометрической декомпозиции расчетной области.
Анализ численной сходимости проведен на задачах, имеющих аналитическое решение.

\defpositions\ соответствуют основным результатам диссертации, приведенным в заключении.

\probation\ Математическая корректность поставленных задач доказана в работах В. С. Владимирова. Построенные численные методы основываются на известных методах коротких и длинных характеристик, корректность которых подтверждена многими авторами. Построенные методы верифицированы на задачах с аналитическим решением.

Основные результаты диссертации изложены в 8 публикациях~\cite{skalko2014, tsybulin2015a, tsybulin2015b, miptconf53,miptconf54,miptconf55,miptconf56,miptconf57},
2 из которых изданы в журналах, рекомендованных ВАК~\cite{skalko2014,tsybulin2015a}%
%, 5 --- в тезисах докладов~\cite{miptconf53,miptconf54,miptconf55,miptconf56,miptconf57}
.

Основные результаты работы докладывались и получили одобрение специалистов на следующих научных конференциях:
\begin{enumerate}
\item 53-й научной конференции МФТИ <<Современные проблемы фундаментальных и прикладных наук>>, Долгопрудный, 2010;
\item 54-й научной конференции МФТИ <<Проблемы фундаментальных и прикладных естественных и технических наук в современном информационном обществе>>, Долгопрудный, 2011;
\item 55-й научной конференции МФТИ <<Проблемы фундаментальных и прикладных естественных и технических наук в современном информационном обществе>>, Долгопрудный, 2012;
\item 56-й научной конференции МФТИ <<Актуальные проблемы фундаментальных и прикладных наук в современном информационном обществе>>, Долгопрудный, 2013;
\item 57-й научной конференции МФТИ <<Актуальные проблемы фундаментальных и прикладных наук в области физики>>, Долгопрудный, 2014.
\end{enumerate}

Основные результаты работы докладывались и получили одобрение специалистов на следующих научных семинарах:
\begin{enumerate}
\item семинар лаборатории математического моделирования нелинейных процессов в газовых средах, МФТИ, Москва, 2012;
\item научная сессия VII школы по высокопроизводительным вычислениям, Университет Иннополис, Казань, 2015;
\item семинар лаборатории флюидодинамики и сейсмоакустики, МФТИ, Москва, 2015;
\item семинар института автоматизации проектирования РАН, Москва, 2015.
\end{enumerate}
